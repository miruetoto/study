\documentclass[12pt,oneside,english]{book}
\usepackage{babel}
\usepackage[utf8]{inputenc}
\usepackage[T1]{fontenc}
\usepackage{color}
\definecolor{marron}{RGB}{60,30,10}
\definecolor{darkblue}{RGB}{0,0,80}
\definecolor{lightblue}{RGB}{80,80,80}
\definecolor{darkgreen}{RGB}{0,80,0}
\definecolor{darkgray}{RGB}{0,80,0}
\definecolor{darkred}{RGB}{80,0,0}
\definecolor{shadecolor}{rgb}{0.97,0.97,0.97}
\usepackage{graphicx}
\usepackage{wallpaper}
\usepackage{wrapfig,booktabs}

\usepackage{fancyhdr}
\usepackage{lettrine}
\input Acorn.fd

\renewcommand{\familydefault}{pplj} 
\usepackage[
final,
stretch=10,
protrusion=true,
tracking=true,
spacing=on,
kerning=on,
expansion=true]{microtype}

%\setlength{\parskip}{1.5ex plus 0.2ex minus 0.2ex}


\usepackage{fourier-orns}
\newcommand{\dash}{\vspace{2em}\noindent \textcolor{darkgray}{\hrulefill~ \raisebox{-2.5pt}[10pt][10pt]{\leafright \decofourleft \decothreeleft  \aldineright \decotwo \floweroneleft \decoone   \floweroneright \decotwo \aldineleft\decothreeright \decofourright \leafleft} ~  \hrulefill \\ \vspace{2em}}}
\newcommand{\rdash}{\noindent \textcolor{darkgray}{ \raisebox{-1.9pt}[10pt][10pt]{\leafright} \hrulefill \raisebox{-1.9pt}[10pt][10pt]{\leafright \decofourleft \decothreeleft  \aldineright \decotwo \floweroneleft \decoone}}}
\newcommand{\ldash}{\textcolor{darkgray}{\raisebox{-1.9pt}[10pt][10pt]{\decoone \floweroneright \decotwo \aldineleft \decothreeright \decofourright \leafleft} \hrulefill \raisebox{-1.9pt}[10pt][10pt]{\leafleft}}}

\fancyhf{}

\renewcommand{\chaptermark}[1]{\markboth{#1}{}}
\renewcommand{\sectionmark}[1]{\markright{#1}}

\newcommand{\estcab}[1]{\itshape\textcolor{marron}{\nouppercase #1}}

\fancyhead[LO]{\estcab{\rightmark}} 
\fancyhead[RO]{\estcab{\leftmark}}
\fancyhead[RO]{\bf\nouppercase{ \leftmark}}
\fancyfoot[RO]{ \leafNE  ~~ \bf \thepage}

\newenvironment{Section}[1]
{\section{\vspace{0ex}#1}}
{\vspace{12pt}\centering ------- \decofourleft\decofourright ------- \par}

\usepackage{lipsum}
\setlength{\parindent}{1em} % Sangría española
\pagestyle{fancy}

\renewcommand{\footnoterule}{\noindent\textcolor{marron}{\decosix \raisebox{2.9pt}{\line(1,0){100}} \lefthand} \vspace{.5em} }
\usepackage[hang,splitrule]{footmisc}
\addtolength{\footskip}{0.5cm}
\setlength{\footnotemargin}{0.3cm}
\setlength{\footnotesep}{0.4cm} 

\usepackage{chngcntr}
\counterwithout{figure}{chapter}
\counterwithout{table}{chapter}

\usepackage{kotex}
\usepackage{amsthm} 
\usepackage{amsmath} 
\usepackage{amsfonts}
\usepackage{enumerate} 
\usepackage{cite}
\usepackage{graphics} 
\usepackage{graphicx,lscape} 
\usepackage{subcaption}
\usepackage{algpseudocode}
\usepackage{algorithm}
\usepackage{titlesec}
\usepackage{cite, url}
\usepackage{amssymb}

\def\ck{\paragraph{\Large$\bullet$}\Large}
\def\goal{\paragraph{\Large(목표)}\Large}
\def\observe{\paragraph{\Large(관찰)}\Large}
\def\assume{\paragraph{\Large(가정)}\Large}
\def\summary{\paragraph{\Large(요약)}\Large}
\def\EX{\paragraph{\Large(예제)}\Large}
\def\guess{\paragraph{\Large(추측)}\Large}
\def\thus{\paragraph{\Large(결론)}\Large}

\def\prob{\paragraph{\Large(문제)}\Large}
\def\sol{\paragraph{\Large(sol)}\Large}
\def\pf{\paragraph{\Large(pf)}\Large}

\def\dfn{\paragraph{\Large(정의)}\Large}
\def\thm{\paragraph{\Large(정리)}\Large}
\def\lem{\paragraph{\Large(레마)}\Large}
\def\promise{\paragraph{\Large(약속)}\Large}
\def\property{\paragraph{\Large(특징)}\Large}
\def\fl{\paragraph{\Large(느낌)}\Large}
\def\memo{\paragraph{\Large(암기)}\Large}

\def\note{\paragraph{\Large\textit{\underline{note:}}}\Large}
\def\ex{\paragraph{\Large\textit{example:}}\Large}


\def\one{\paragraph{\Large(1)}\Large}
\def\two{\paragraph{\Large(2)}\Large}
\def\three{\paragraph{\Large(3)}\Large}
\def\four{\paragraph{\Large(4)}\Large}
\def\five{\paragraph{\Large(5)}\Large}
\def\six{\paragraph{\Large(6)}\Large}
\def\seven{\paragraph{\Large(7)}\Large}
\def\eight{\paragraph{\Large(8)}\Large}
\def\nine{\paragraph{\Large(9)}\Large}
\def\ten{\paragraph{\Large(10)}\Large}

\def\cka{\paragraph{\Large(a)}\Large}
\def\ckb{\paragraph{\Large(b)}\Large}
\def\ckc{\paragraph{\Large(c)}\Large}
\def\ckd{\paragraph{\Large(d)}\Large}
\def\cke{\paragraph{\Large(e)}\Large}
\def\ckf{\paragraph{\Large(f)}\Large}
\def\ckg{\paragraph{\Large(g)}\Large}
\def\ckh{\paragraph{\Large(h)}\Large}
\def\cki{\paragraph{\Large(i)}\Large}
\def\ckj{\paragraph{\Large(j)}\Large}

\newcommand{\bld}[1]{\mbox{\boldmath $#1$}}
\newcommand{\bs}[1]{\mbox{\boldmath $#1$}}

\newcommand{\bsa}{\mbox{\boldmath $a$}}
\newcommand{\bsb}{\mbox{\boldmath $b$}}
\newcommand{\bsc}{\mbox{\boldmath $c$}}
\newcommand{\bsd}{\mbox{\boldmath $d$}}
\newcommand{\bse}{\mbox{\boldmath $e$}}
\newcommand{\bsf}{\mbox{\boldmath $f$}}
\newcommand{\bsg}{\mbox{\boldmath $g$}}
\newcommand{\bsh}{\mbox{\boldmath $h$}}
\newcommand{\bsi}{\mbox{\boldmath $i$}}
\newcommand{\bsj}{\mbox{\boldmath $j$}}
\newcommand{\bsk}{\mbox{\boldmath $k$}}
\newcommand{\bsl}{\mbox{\boldmath $l$}}
\newcommand{\bsm}{\mbox{\boldmath $m$}}
\newcommand{\bsn}{\mbox{\boldmath $n$}}
\newcommand{\bso}{\mbox{\boldmath $o$}}
\newcommand{\bsp}{\mbox{\boldmath $p$}}
\newcommand{\bsq}{\mbox{\boldmath $q$}}
\newcommand{\bsr}{\mbox{\boldmath $r$}}
\newcommand{\bss}{\mbox{\boldmath $s$}}
\newcommand{\bst}{\mbox{\boldmath $t$}}
\newcommand{\bsu}{\mbox{\boldmath $u$}}
\newcommand{\bsv}{\mbox{\boldmath $v$}}
\newcommand{\bsw}{\mbox{\boldmath $w$}}
\newcommand{\bsx}{\mbox{\boldmath $x$}}
\newcommand{\bsy}{\mbox{\boldmath $y$}}
\newcommand{\bsz}{\mbox{\boldmath $z$}}

\newcommand{\bfa}{\mbox{$\bf{a}$}}
\newcommand{\bfb}{\mbox{$\bf{b}$}}
\newcommand{\bfc}{\mbox{$\bf{c}$}}
\newcommand{\bfd}{\mbox{$\bf{d}$}}
\newcommand{\bfe}{\mbox{$\bf{e}$}}
\newcommand{\bff}{\mbox{$\bf{f}$}}
\newcommand{\bfg}{\mbox{$\bf{g}$}}
\newcommand{\bfh}{\mbox{$\bf{h}$}}
\newcommand{\bfi}{\mbox{$\bf{i}$}}
\newcommand{\bfj}{\mbox{$\bf{j}$}}
\newcommand{\bfk}{\mbox{$\bf{k}$}}
\newcommand{\bfl}{\mbox{$\bf{l}$}}
\newcommand{\bfm}{\mbox{$\bf{m}$}}
\newcommand{\bfn}{\mbox{$\bf{n}$}}
\newcommand{\bfo}{\mbox{$\bf{o}$}}
\newcommand{\bfp}{\mbox{$\bf{p}$}}
\newcommand{\bfq}{\mbox{$\bf{q}$}}
\newcommand{\bfr}{\mbox{$\bf{r}$}}
\newcommand{\bfs}{\mbox{$\bf{s}$}}
\newcommand{\bft}{\mbox{$\bf{t}$}}
\newcommand{\bfu}{\mbox{$\bf{u}$}}
\newcommand{\bfv}{\mbox{$\bf{v}$}}
\newcommand{\bfw}{\mbox{$\bf{w}$}}
\newcommand{\bfx}{\mbox{$\bf{x}$}}
\newcommand{\bfy}{\mbox{$\bf{y}$}}
\newcommand{\bfz}{\mbox{$\bf{z}$}}

\newcommand{\bsA}{\mbox{$\boldmath{A}$}}
\newcommand{\bsB}{\mbox{$\boldmath{B}$}}
\newcommand{\bsC}{\mbox{$\boldmath{C}$}}
\newcommand{\bsD}{\mbox{$\boldmath{D}$}}
\newcommand{\bsE}{\mbox{$\boldmath{E}$}}
\newcommand{\bsF}{\mbox{$\boldmath{F}$}}
\newcommand{\bsG}{\mbox{$\boldmath{G}$}}
\newcommand{\bsH}{\mbox{$\boldmath{H}$}}
\newcommand{\bsI}{\mbox{$\boldmath{I}$}}
\newcommand{\bsJ}{\mbox{$\boldmath{J}$}}
\newcommand{\bsK}{\mbox{$\boldmath{K}$}}
\newcommand{\bsL}{\mbox{$\boldmath{L}$}}
\newcommand{\bsM}{\mbox{$\boldmath{M}$}}
\newcommand{\bsN}{\mbox{$\boldmath{N}$}}
\newcommand{\bsO}{\mbox{$\boldmath{O}$}}
\newcommand{\bsP}{\mbox{$\boldmath{P}$}}
\newcommand{\bsQ}{\mbox{$\boldmath{Q}$}}
\newcommand{\bsR}{\mbox{$\boldmath{R}$}}
\newcommand{\bsS}{\mbox{$\boldmath{S}$}}
\newcommand{\bsT}{\mbox{$\boldmath{T}$}}
\newcommand{\bsU}{\mbox{$\boldmath{U}$}}
\newcommand{\bsV}{\mbox{$\boldmath{V}$}}
\newcommand{\bsW}{\mbox{$\boldmath{W}$}}
\newcommand{\bsX}{\mbox{$\boldmath{X}$}}
\newcommand{\bsY}{\mbox{$\boldmath{Y}$}}
\newcommand{\bsZ}{\mbox{$\boldmath{Z}$}}

\newcommand{\bfA}{\mbox{$\bf{A}$}}
\newcommand{\bfB}{\mbox{$\bf{B}$}}
\newcommand{\bfC}{\mbox{$\bf{C}$}}
\newcommand{\bfD}{\mbox{$\bf{D}$}}
\newcommand{\bfE}{\mbox{$\bf{E}$}}
\newcommand{\bfF}{\mbox{$\bf{F}$}}
\newcommand{\bfG}{\mbox{$\bf{G}$}}
\newcommand{\bfH}{\mbox{$\bf{H}$}}
\newcommand{\bfI}{\mbox{$\bf{I}$}}
\newcommand{\bfJ}{\mbox{$\bf{J}$}}
\newcommand{\bfK}{\mbox{$\bf{K}$}}
\newcommand{\bfL}{\mbox{$\bf{L}$}}
\newcommand{\bfM}{\mbox{$\bf{M}$}}
\newcommand{\bfN}{\mbox{$\bf{N}$}}
\newcommand{\bfO}{\mbox{$\bf{O}$}}
\newcommand{\bfP}{\mbox{$\bf{P}$}}
\newcommand{\bfQ}{\mbox{$\bf{Q}$}}
\newcommand{\bfR}{\mbox{$\bf{R}$}}
\newcommand{\bfS}{\mbox{$\bf{S}$}}
\newcommand{\bfT}{\mbox{$\bf{T}$}}
\newcommand{\bfU}{\mbox{$\bf{U}$}}
\newcommand{\bfV}{\mbox{$\bf{V}$}}
\newcommand{\bfW}{\mbox{$\bf{W}$}}
\newcommand{\bfX}{\mbox{$\bf{X}$}}
\newcommand{\bfY}{\mbox{$\bf{Y}$}}
\newcommand{\bfZ}{\mbox{$\bf{Z}$}}

\DeclareMathOperator*{\argmin}{argmin} %\usepackage{amsmath}를 써야지 정의가능함. 
\DeclareMathOperator*{\argmax}{argmax} %\usepackage{amsmath}를 써야지 정의가능함. 

\usepackage{titlesec}
\titleformat*{\section}{\huge\bfseries}
\titleformat*{\subsection}{\huge\bfseries}
\titleformat*{\subsubsection}{\huge\bfseries}
\titleformat*{\paragraph}{\huge\bfseries}
\titleformat*{\subparagraph}{\huge\bfseries}


\titleclass{\part}{top}
\titleformat{\part}[display]
  {\normalfont\huge\bfseries}{\centering\partname\ \thepart}{20pt}{\Huge\centering}
\titlespacing*{\part}{0pt}{50pt}{40pt}
\titleclass{\chapter}{straight}
\titleformat{\chapter}[display]
  {\normalfont\huge\bfseries}{\chaptertitlename\ \thechapter}{20pt}{\Huge}
\titlespacing*{\chapter} {0pt}{50pt}{40pt}

\newcommand*\initfamily{\usefont{U}{Acorn}{xl}{n}}
\usepackage[left=10px,right=10px,top=10px,bottom=10px,paperwidth=8in,paperheight=32in]{geometry}

\usepackage{geometry}
\geometry{
tmargin=3cm, 
bmargin=3cm, 
lmargin=1cm, 
rmargin=1cm,
headheight=1.5cm,
headsep=0.8cm,
footskip=0.5cm}
\linespread{1}
\begin{document}

\subsubsection{백분위수}

\ck 이번에는 백분위수에 대하여 다룬다. 
\ck 참고교재는 김우철 수리통계학책. 

\dash
\subsubsection{정리 4.4.3} 
\ck $X$가 연속형 확률변수라고 하자. $F(x)$를 $X$의 \emph{cdf}라고 하자. 

\ck $F(x)$도 분명히 함수이므로 $F(X)$를 정의할 수 있다. $F(X)$는 확률변수가 된다. 

\ck 확률변수이므로 분포를 따른텐데 $F(X)$는 아래의 분포를 따름이 알려져 있다. 
\[
F(X) \sim U(0,1)
\]

\ck 또한 $F^{-1}(U)$ 역시 확률변수가 된다. (단 $U\sim U(0,1)$.) 그런데 이 확률변수는 $F^{-1}(U)$ 는 $X$와 분포가 같음이 알려져 있다. 즉 
\[
F^{-1}(U)\overset{d}{=}X 
\]
이다. 

\rdash 

\ck $U_1,\dots,U_n \overset{iid}{\sim} U(0,1)$ 라고 하자. 그리고 $U_{(1)},\dots,U_{(n)}$을 $U_1,\dots,U_n$의 순서통계량이라고 하자. 정리 4.4.3 에 의해서 
\begin{align*}
F^{-1}(U_{(1)}):=X_1^* & \sim F \\
F^{-1}(U_{(2)}):=X_2^* & \sim F \\
& \dots \\ 
F^{-1}(U_{(n)}):=X_n^* & \sim F
\end{align*}
이다. 

\ck $F$가 순증가 함수이므로 $X_1^*<\dots<X_n^*$이다. 

\ck 아래가 성립한다. (왜??)
\[
\begin{pmatrix}
X_1^* \\ \dots \\ X_n^*
\end{pmatrix}\overset{d}{=}
\begin{pmatrix}
X_{(1)} \\ \dots \\ X_{(n)}
\end{pmatrix}
\]
이는 $X$의 cdf $F$를 알고 있을 경우 $X$의 순서통계량을 어떻게 생성할지 알려준다. 

\subsubsection{정리 4.4.3.} 

\ck 어떤분포의 순서통계량: $X_1,\dots,X_n \overset{iid}{\sim} F ~ \Longrightarrow ~ X_{(1)},\dots,X_{(r)}$.
\ck 균등분포의 순서통계량: $U_1,\dots,U_n \overset{iid}{\sim} U(0,1) ~ \Longrightarrow ~ U_{(1)},\dots,U_{(n)}$. 

\ck 지수분포의 순서통계량: $Y_1,\dots,Y_n \overset{iid}{\sim} Exp(1) ~ \Longrightarrow ~ Y_{(1)},\dots,Y_{(n)}$. 

\ck 균등분포의 순서통계량과 지수분포의 순서통계량에는 아래와 같은 관계가 있다. (예제 4.3.4.)
\[
U_{(r)}\overset{d}{=}1-e^{-Y_{(r)}}, \quad r=1,2,\dots,n
\]

\ck 그런데 임의의 $r=1,2,\dots,n$에 대하여 지수분포의 순서통계량 $Y_{(r)}$은 아래를 만족한다. (예제 4.3.3.)
\[
\forall r,~  \exists Z_1,\dots,Z_r \overset{iid}{\sim} Exp(1) ~ s.t. \quad Y_{(r)}\overset{d}{=}\frac{1}{n}Z_1+\dots+\frac{1}{n-r+1}Z_r
\]

\rdash 

\ck 순서통계량 $X_{(r)}$은 아래와 같이 얻을 수 있다. 
\[
X_{(r)}\overset{d}{=}F^{-1}(U_{(r)})
\]
\ck 그런데 (1) $U_{(r)}\overset{d}{=}1-e^{-Y_{(r)}}$ 와 (2) $Y_{(r)}\overset{d}{=}\frac{1}{n}Z_1+\dots+\frac{1}{n-r+1}Z_r$을 이용하면 
\[
X_{(r)}\overset{d}{=}F^{-1}\left(1-e^{-(\frac{1}{n}Z_1+\dots+\frac{1}{n-r+1}Z_r)}\right)
\]
따라서 $h(\bigstar)=F^{-1}\big(1-e^{-\bigstar}\big)$라고 정의하면 
\[
X_{(r)}\overset{d}{=}h\left(\frac{1}{n}Z_1+\dots+\frac{1}{n-r+1}Z_r\right)
\]
가 성립한다. 

\subsubsection{연습문제 5.16.}

\ck $r_n \sim \alpha n \Longleftrightarrow r_n/n \to \alpha$. 
\ck $s_n \sim \beta n \Longleftrightarrow r_n/n \to \beta$. 
\note 위의정의는 예제 5.2.7 에 나와있다. 
\ck 그리고 $0<\alpha<\beta<1$. 
\ck 편의상 아래를 가정하자. 
\one $r_n$을 그냥 $r$로 쓰자. 여기에서 $r$은 $n$개의 sample 중 하위25\%에 번째에 해당하는 수 이다. 즉 $r/n \approx 1/4$. 동일한 논리로 $s_n$도 그냥 $s$라고 쓰자. 
\two $r/n \to \frac{1}{4}$ and $s/n \to \frac{3}{4}$. 
\three $\alpha,\beta$를 그냥 $\alpha_r,\alpha_s$로 정의하자. 이는 분위수의 느낌을 좀더 강조하기 위해서이다. $\alpha_r$은 $r$에 해당하는 분위수라는 뜻임. 

\ck $X_{(r)}$과 $X_{(s)}$의 join pdf를 구하라. 

\sol 

\ck 우선 순서통계량 문제이므로 아래와 같은 확률변수를 가정하자. 
\[
Y_1,\dots,Y_n \overset{iid}{\sim} Exp(1).
\]
\ck 아래와 같은 벡터를 가정하자. 
\[
\begin{bmatrix}Y_{(r)} \\ Y_{(s)}\end{bmatrix}=\begin{bmatrix} \frac{1}{n}Z_1+\dots+\frac{1}{n-r+1}Z_r \\  \frac{1}{n}Z_1+\dots+\frac{1}{n-s+1}Z_s \end{bmatrix}
\]

\ck 전체적인 그림.
\begin{align*}
& \mbox{$Y_{(r)}$과 $Y_{(s)}$번째 순서통계량의 분포} \\ 
& \Longrightarrow X_1,\dots,X_n \overset{iid}{\sim} F \quad \mbox{인 임의의 분포에서 $X_{(r)}$과 $X_{(s)}$의 분포}
\end{align*}


\ck 아래가 성립한다. (예제 5.2.7)
\begin{align*}
& E(Y_{(r)})=E\left(\frac{1}{n}Z_1+\dots+\frac{1}{n-r+1}Z_r\right)=
\frac{1}{n}+\dots+\frac{1}{n-r+1} \\ 
& =\frac{1}{n} \sum_{k=1}^{r-1}\frac{1}{1-k/n} \approx \int_0^{\alpha_r}\frac{1}{1-x}dx=-\log(1-\alpha_r)
\end{align*}

\ck 아래도 성립한다. (예제 5.2.7)
\begin{align*}
&V(Y_{(r)})= V\left(\frac{1}{n}Z_1+\dots+\frac{1}{n-r+1}Z_r\right)=
\frac{1}{n^2}+\dots+\frac{1}{(n-r+1)^2} \\ 
& =\frac{1}{n^2} \sum_{k=1}^{r-1}\frac{1}{(1-k/n)^2} \approx \int_0^{\alpha_r}\frac{1}{(1-x)^2}dx=\frac{1}{n}\frac{\alpha_r}{1-\alpha_r}
\end{align*}

\ck 따라서 
\[
\sqrt{n}\left(\begin{bmatrix}Y_{(r)} \\ Y_{(s)}\end{bmatrix}-\begin{bmatrix}-\log(1-\alpha_r) \\ -\log(1-\alpha_s)\end{bmatrix}\right)\overset{d}{\to} N\left(\begin{bmatrix}0\\0\end{bmatrix},\begin{bmatrix} \frac{\alpha_r}{1-\alpha_r} & \frac{\alpha_r}{1-\alpha_r} \\ \frac{\alpha_r}{1-\alpha_r} & \frac{\alpha_s}{1-\alpha_s} \end{bmatrix}\right).
\]


\ck 그런데 
\[
\begin{bmatrix} X_{(r)}\\ X_{(s)} \end{bmatrix}=\begin{bmatrix} h(Y_{(r)})\\  h(Y_{(s)}) \end{bmatrix}=\bsh \left(\begin{bmatrix}Y_{(r)} \\ Y_{(s)}\end{bmatrix}\right)
\]

\ck 따라서 
\begin{align*}
& \sqrt{n}\times\left(\begin{bmatrix}X_{(r)} \\ X_{(s)}\end{bmatrix}-\begin{bmatrix}? \\ ? \end{bmatrix}\right) \\ 
& =\sqrt{n}\times\left(\bsh\left(\begin{bmatrix}Y_{(r)} \\ Y_{(s)}\end{bmatrix}\right)-\bsh\left(\begin{bmatrix} -\log(1-\alpha_r) \\ -\log(1-\alpha_s) \end{bmatrix}\right)\right) \\
& =\sqrt{n}\times\left(\begin{bmatrix}h(Y_{(r)}) \\ h(Y_{(s)})\end{bmatrix}-\begin{bmatrix} h(-\log(1-\alpha_r)) \\ h(-\log(1-\alpha_s)) \end{bmatrix}\right)\\ 
& \overset{d}{\longrightarrow} 
\begin{bmatrix}
?? & ?? \\
?? & ??
\end{bmatrix}^T N\left(\begin{bmatrix}0\\0\end{bmatrix},\begin{bmatrix} \frac{\alpha_r}{1-\alpha_r} & \frac{\alpha_r}{1-\alpha_r} \\ \frac{\alpha_r}{1-\alpha_r} & \frac{\alpha_s}{1-\alpha_s} \end{bmatrix}\right)
\end{align*}

\note 여기에서 $\bsh: \mathbb{R}^2 \to \mathbb{R}^2$ 이므로 $\begin{bmatrix}?? & ?? \\?? & ??\end{bmatrix}$와 같이 $2\times 2$매트릭스가 나왔다. 
\note 만약에 $\bsh: \mathbb{R}^2 \to \mathbb{R}^3$이었다면 즉 $h(x_1,x_2)=(y_1,y_2,y_3)$ 꼴이었다면 
\[
\begin{bmatrix}
\frac{\partial }{\partial x_1} \\ \frac{\partial }{\partial x_2}\end{bmatrix} \begin{bmatrix} y_1 & y_2 & y_3\end{bmatrix}=
\begin{bmatrix}
\frac{\partial y_1 }{\partial y_1} & \frac{\partial y_2}{\partial x_1} & \frac{\partial y_3}{\partial x_1} \\ 
\frac{\partial y_1}{\partial x_2} & \frac{\partial y_2}{\partial x_2} & \frac{\partial y_3}{\partial x_2} 
\end{bmatrix}
\]
일것이다. 

\rdash 

\ck 잠시 시간을 투자하여 $\begin{bmatrix}?? & ?? \\?? & ??\end{bmatrix}$를 계산하자. 
\[
\begin{bmatrix}?? & ?? \\?? & ??\end{bmatrix}
=
\begin{bmatrix}\frac{\partial h(x_1)}{\partial x_1} & \frac{\partial h(x_2)}{\partial x_1} \\ \frac{\partial h(x_1)}{\partial x_2} & \frac{\partial h(x_2)}{\partial x_2} \end{bmatrix}
\] 
이다. 그런데 
\[
h(x_1)=F^{-1}(1-e^{-x_1})
\]
이므로 
\[
F\big(h(x_1)\big)=1-e^{-x_1}
\]
그러므로 
\[
\frac{\partial}{\partial x_1} F(h(x_1))=e^{-x_1}
\]
따라서 
\[
f(h(x_1))\frac{\partial h(x_1)}{\partial x_1}=e^{-x_1} \Longleftrightarrow \frac{\partial h(x_1)}{\partial x_1}=\frac{e^{-x_1}}{f(h(x_1))}
\]
$x_1$대신에 $-\log(1-\alpha_r)$를 대입하자. 
\[
e^{-x_1}=e^{\log(1-\alpha_r)}= 1-\alpha_r
\]
\[
f(h(x_1))=f(h(-\log(1-\alpha_r)))=f\big(F^{-1}(1-e^{\log(1-\alpha_r)})\big)=f \circ F^{-1}(\alpha_r)
\]
따라서 
\[
\begin{bmatrix}?? & ?? \\ ?? & ??\end{bmatrix}=
\begin{bmatrix}\frac{1-\alpha_r}{f\circ F^{-1}(\alpha_r)} & 0 \\ 0 & \frac{1-\alpha_s}{f\circ F^{-1}(\alpha_s)}\end{bmatrix}
\]
\note $X$가 연속확률변수이므로 $F^{-1}$도 연속함수이다. 따라서 
\[
\begin{bmatrix}\frac{1-\alpha_r}{f\circ F^{-1}(\alpha_r)} & 0 \\ 0 & \frac{1-\alpha_s}{f\circ F^{-1}(\alpha_s)}\end{bmatrix}
\]
의 각 원소가 모두 연속이다. 따라서 모든 편미분이 존재하고 그것이 연속이다. 

\rdash 

\ck 따라서 수렴하는 분포는 
\[
\begin{bmatrix}\frac{1-\alpha_r}{f\circ F^{-1}(\alpha_r)} & 0 \\ 0 & \frac{1-\alpha_s}{f\circ F^{-1}(\alpha_s)}\end{bmatrix}^T N\left(\begin{bmatrix}0\\0\end{bmatrix},\begin{bmatrix} \frac{\alpha_r}{1-\alpha_r} & \frac{\alpha_r}{1-\alpha_r} \\ \frac{\alpha_r}{1-\alpha_r} & \frac{\alpha_s}{1-\alpha_s} \end{bmatrix}\right)
\]
와 같다. 따라서 수렴하는 분포의 분산은 
\begin{align*}
& \begin{bmatrix}\frac{1-\alpha_r}{f\circ F^{-1}(\alpha_r)} & 0 \\ 0 & \frac{1-\alpha_s}{f\circ F^{-1}(\alpha_s)}\end{bmatrix}^T 
\begin{bmatrix} \frac{\alpha_r}{1-\alpha_r} & \frac{\alpha_r}{1-\alpha_r} \\ \frac{\alpha_r}{1-\alpha_r} & \frac{\alpha_s}{1-\alpha_s} \end{bmatrix}
\begin{bmatrix}\frac{1-\alpha_r}{f\circ F^{-1}(\alpha_r)} & 0 \\ 0 & \frac{1-\alpha_s}{f\circ F^{-1}(\alpha_s)}\end{bmatrix} \\
& = \begin{bmatrix}
\frac{1-\alpha_r}{f\circ F^{-1}(\alpha_r)} & 0 \\ 0 & \frac{1-\alpha_s}{f\circ F^{-1}(\alpha_s)}
\end{bmatrix}
\begin{bmatrix}
\frac{\alpha_r}{1-\alpha_r}\frac{1-\alpha_r}{f\circ F^{-1}(\alpha_r)} & \frac{\alpha_r}{1-\alpha_r}\frac{1-\alpha_s}{f\circ F^{-1}(\alpha_s)} \\ 
\frac{\alpha_r}{1-\alpha_r}\frac{1-\alpha_r}{f\circ F^{-1}(\alpha_r)} & 
\frac{\alpha_s}{1-\alpha_s}\frac{1-\alpha_s}{f\circ F^{-1}(\alpha_s)}
\end{bmatrix} \\ 
& =
\begin{bmatrix}
\frac{1-\alpha_r}{f\circ F^{-1}(\alpha_r)}\frac{\alpha_r}{1-\alpha_r}\frac{1-\alpha_r}{f\circ F^{-1}(\alpha_r)} & \frac{1-\alpha_r}{f\circ F^{-1}(\alpha_r)}\frac{\alpha_r}{1-\alpha_r}\frac{1-\alpha_s}{f\circ F^{-1}(\alpha_s)} \\ 
\frac{1-\alpha_s}{f\circ F^{-1}(\alpha_s)}\frac{\alpha_r}{1-\alpha_r}\frac{1-\alpha_r}{f\circ F^{-1}(\alpha_r)} & 
\frac{1-\alpha_s}{f\circ F^{-1}(\alpha_s)}\frac{\alpha_s}{1-\alpha_s}\frac{1-\alpha_s}{f\circ F^{-1}(\alpha_s)}
\end{bmatrix}\\
& =
\begin{bmatrix}
\frac{(1-\alpha_r)\alpha_r}{(f\circ F^{-1}(\alpha_r))^2} & \frac{\alpha_r(1-\alpha_s)}{f\circ F^{-1}(\alpha_r) \times f\circ F^{-1}(\alpha_s)} \\ 
\frac{\alpha_r(1-\alpha_s)}{f\circ F^{-1}(\alpha_r) \times f\circ F^{-1}(\alpha_s)} & 
\frac{(1-\alpha_s)\alpha_s}{(f\circ F^{-1}(\alpha_s))^2} 
\end{bmatrix}
\end{align*}

\ck 결론적으로 아래와 같이 주장할 수 있다. 
\begin{align*}
& \sqrt{n}\times\left(\begin{bmatrix}X_{(r)} \\ X_{(s)}\end{bmatrix}-\begin{bmatrix}? \\ ? \end{bmatrix}\right) \\
& \overset{d}{\to} 
N\left(\begin{bmatrix}0\\0\end{bmatrix},\begin{bmatrix}
\frac{(1-\alpha_r)\alpha_r}{(f\circ F^{-1}(\alpha_r))^2} & \frac{\alpha_r(1-\alpha_s)}{f\circ F^{-1}(\alpha_r) \times f\circ F^{-1}(\alpha_s)} \\ 
\frac{\alpha_r(1-\alpha_s)}{f\circ F^{-1}(\alpha_r) \times f\circ F^{-1}(\alpha_s)} & 
\frac{(1-\alpha_s)\alpha_s}{(f\circ F^{-1}(\alpha_s))^2} 
\end{bmatrix}\right)
\end{align*}
여기에서 
\begin{align*}
&\begin{bmatrix}? \\ ? \end{bmatrix}=
\begin{bmatrix}h(EY_{(r)}) \\ h(EY_{(s)}) \end{bmatrix}=\begin{bmatrix}h(-\log(1-\alpha_r)) \\ h(-\log(1-\alpha_s)) \end{bmatrix} \\
&=\begin{bmatrix}F^{-1}\circ \big(1-e^{\log(1-\alpha_r)}
\big) \\ F^{-1}\circ \big(1-e^{\log(1-\alpha_r)}\big)  \end{bmatrix} 
= \begin{bmatrix} F^{-1}(\alpha_r) \\ F^{-1}(\alpha_s) \end{bmatrix}
\end{align*}
따라서 
\begin{align*}
& \sqrt{n}\times\left(\begin{bmatrix}X_{(r)} \\ X_{(s)}\end{bmatrix}-\begin{bmatrix} F^{-1}(\alpha_r) \\ F^{-1}(\alpha_s) \end{bmatrix}\right) \\
& \overset{d}{\to} 
N\left(\begin{bmatrix} 0 \\ 0 \end{bmatrix},\begin{bmatrix}
\frac{(1-\alpha_r)\alpha_r}{(f\circ F^{-1}(\alpha_r))^2} & \frac{\alpha_r(1-\alpha_s)}{f\circ F^{-1}(\alpha_r) \times f\circ F^{-1}(\alpha_s)} \\ 
\frac{\alpha_r(1-\alpha_s)}{f\circ F^{-1}(\alpha_r) \times f\circ F^{-1}(\alpha_s)} & 
\frac{(1-\alpha_s)\alpha_s}{(f\circ F^{-1}(\alpha_s))^2} 
\end{bmatrix}\right)
\end{align*}

\ck 마지날을 구해보자. 
\[
\sqrt{n}\big(X_{(r)}-F^{-1}(\alpha_r)\big) \overset{d}{\to} N\left(0, \frac{(1-\alpha_r)\alpha_r}{(f\circ F^{-1}(\alpha_r))^2} \right)
\]
\[
\sqrt{n}\big(X_{(s)}-F^{-1}(\alpha_s)\big) \overset{d}{\to} N\left(0, \frac{(1-\alpha_s)\alpha_s}{(f\circ F^{-1}(\alpha_s))^2} \right)
\]

\note 참고로 $F^{-1}(\alpha_r)$는 $\alpha_r$-percentile의 정의가 된다. 그리고 $X_{(r)}$는 sample $\alpha_r$-percentile 이라 볼 수 있다. 

\note 따라서 위의 식을 관찰하면 표본백분위수는 백분위수로 확률수렴함을 알 수 있다. 

\note 추가적으로 표본 백분위수의 분포도 알 수 있다. 

\note 또한 아래와 같이 범위(백분위수간의 차이)의 분포도 알 수 있다. 

\ck $R=X_{(s)}-X_{(r)}$의 분포를 구해보자. 정규분포의 차는 다시 정규분포를 따르므로 
\[
\sqrt{n}(R-\mu) \overset{d}{\to} N(0,\sigma^2)
\]
여기에서 
\[
\mu=F^{-1}(\alpha_s)-F^{-1}(\alpha_r)
\]
\[
\sigma^2=\frac{(1-\alpha_s)\alpha_s}{(f\circ F^{-1}(\alpha_s))^2}+\frac{(1-\alpha_r)\alpha_r}{(f\circ F^{-1}(\alpha_r))^2}+2 \frac{\alpha_r(1-\alpha_s)}{f\circ F^{-1}(\alpha_r) \times f\circ F^{-1}(\alpha_s)}
\]

\ck 위의식에서 $\alpha_r=\frac{1}{4}$, $\alpha_s=\frac{3}{4}$를 대입하면 
\[
\mu=F^{-1}(3/4)-F^{-1}(1/4)
\]
\begin{align*}
&\sigma^2=\frac{(1-3/4)3/4}{(f\circ F^{-1}(3/4))^2}+\frac{(1-1/4)1/4}{(f\circ F^{-1}(1/4))^2}+2 \frac{1/4(1-3/4)}{f\circ F^{-1}(1/4) \times f\circ F^{-1}(3/4)}\\
&=\frac{1}{16}\left(\frac{3}{(f\circ F^{-1}(3/4))^2}+\frac{3}{(f\circ F^{-1}(1/4))^2}+ \frac{2}{f\circ F^{-1}(1/4) \times f\circ F^{-1}(3/4)}\right)
\end{align*}


\end{document}
