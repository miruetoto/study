\documentclass[12pt,oneside,english,a4paper]{article}
\usepackage{babel}
\usepackage[utf8]{inputenc}
\usepackage[T1]{fontenc}
\usepackage{color}
\usepackage{graphicx}
\usepackage{wallpaper}
\usepackage{wrapfig,booktabs}

\usepackage{fancyhdr}

\usepackage{fourier-orns}
\newcommand{\dash}{\noindent \newline\textcolor{black}{\hrulefill~ \raisebox{-2.5pt}[10pt][10pt]{\leafright \decofourleft \decothreeleft  \aldineright \decotwo \floweroneleft \decoone   \floweroneright \decotwo \aldineleft\decothreeright \decofourright \leafleft} ~  \hrulefill}}

\usepackage{titlesec}
\titleformat*{\section}{\it\huge\bfseries}
\titleformat*{\subsection}{\it\huge\bfseries}
\titleformat*{\subsubsection}{\it\LARGE\bfseries}
\titleformat*{\paragraph}{\huge\bfseries}
\titleformat*{\subparagraph}{\LARGE\bfseries}

\usepackage[left=20px,right=20px,top=50px,bottom=50px,paperwidth=8in,paperheight=12in]{geometry}


\usepackage[cjk]{kotex}

\usepackage{amsthm} 
\usepackage{amsmath} 
\usepackage{amsfonts}
\usepackage{enumerate} 
\usepackage{cite}
\usepackage{graphics} 
\usepackage{graphicx,lscape} 
\usepackage{subcaption}
\usepackage{algpseudocode}
\usepackage{algorithm}
\usepackage{titlesec}
\usepackage{cite, url}
\usepackage{amssymb}

\def\bk{\paragraph{\large$$}\large}
\def\ck{\paragraph{\large$\bullet$}\large}
\def\pf{\paragraph{\large(pf)}\large}
\def\note{\paragraph{\large\textit{\underline{note:}}}\large}
\def\ex{\paragraph{\large\textit{example:}}\large}
\newcommand{\para}[1]{\paragraph{\large\it\underline{\textbf{#1:}}}\large}
\newcommand{\parablue}[1]{\paragraph{\large\textcolor{blue}{\it\underline{\textbf{#1:}}}}\large}
\newcommand{\parared}[1]{\paragraph{\large\textcolor{red}{\it\underline{\textbf{#1:}}}}\large}
\newcommand{\paraviolet}[1]{\paragraph{\large\textcolor{violet}{\it\underline{\textbf{#1:}}}}\large}
\newcommand{\paraorange}[1]{\paragraph{\large\textcolor{orange}{\it\underline{\textbf{#1:}}}}\large}



\newcommand{\bs}[1]{\mbox{\boldmath $#1$}}

\newcommand{\bsa}{\mbox{\boldmath $a$}}
\newcommand{\bsb}{\mbox{\boldmath $b$}}
\newcommand{\bsc}{\mbox{\boldmath $c$}}
\newcommand{\bsd}{\mbox{\boldmath $d$}}
\newcommand{\bse}{\mbox{\boldmath $e$}}
\newcommand{\bsf}{\mbox{\boldmath $f$}}
\newcommand{\bsg}{\mbox{\boldmath $g$}}
\newcommand{\bsh}{\mbox{\boldmath $h$}}
\newcommand{\bsi}{\mbox{\boldmath $i$}}
\newcommand{\bsj}{\mbox{\boldmath $j$}}
\newcommand{\bsk}{\mbox{\boldmath $k$}}
\newcommand{\bsl}{\mbox{\boldmath $l$}}
\newcommand{\bsm}{\mbox{\boldmath $m$}}
\newcommand{\bsn}{\mbox{\boldmath $n$}}
\newcommand{\bso}{\mbox{\boldmath $o$}}
\newcommand{\bsp}{\mbox{\boldmath $p$}}
\newcommand{\bsq}{\mbox{\boldmath $q$}}
\newcommand{\bsr}{\mbox{\boldmath $r$}}
\newcommand{\bss}{\mbox{\boldmath $s$}}
\newcommand{\bst}{\mbox{\boldmath $t$}}
\newcommand{\bsu}{\mbox{\boldmath $u$}}
\newcommand{\bsv}{\mbox{\boldmath $v$}}
\newcommand{\bsw}{\mbox{\boldmath $w$}}
\newcommand{\bsx}{\mbox{\boldmath $x$}}
\newcommand{\bsy}{\mbox{\boldmath $y$}}
\newcommand{\bsz}{\mbox{\boldmath $z$}}

\newcommand{\bfa}{\mbox{$\bf{a}$}}
\newcommand{\bfb}{\mbox{$\bf{b}$}}
\newcommand{\bfc}{\mbox{$\bf{c}$}}
\newcommand{\bfd}{\mbox{$\bf{d}$}}
\newcommand{\bfe}{\mbox{$\bf{e}$}}
\newcommand{\bff}{\mbox{$\bf{f}$}}
\newcommand{\bfg}{\mbox{$\bf{g}$}}
\newcommand{\bfh}{\mbox{$\bf{h}$}}
\newcommand{\bfi}{\mbox{$\bf{i}$}}
\newcommand{\bfj}{\mbox{$\bf{j}$}}
\newcommand{\bfk}{\mbox{$\bf{k}$}}
\newcommand{\bfl}{\mbox{$\bf{l}$}}
\newcommand{\bfm}{\mbox{$\bf{m}$}}
\newcommand{\bfn}{\mbox{$\bf{n}$}}
\newcommand{\bfo}{\mbox{$\bf{o}$}}
\newcommand{\bfp}{\mbox{$\bf{p}$}}
\newcommand{\bfq}{\mbox{$\bf{q}$}}
\newcommand{\bfr}{\mbox{$\bf{r}$}}
\newcommand{\bfs}{\mbox{$\bf{s}$}}
\newcommand{\bft}{\mbox{$\bf{t}$}}
\newcommand{\bfu}{\mbox{$\bf{u}$}}
\newcommand{\bfv}{\mbox{$\bf{v}$}}
\newcommand{\bfw}{\mbox{$\bf{w}$}}
\newcommand{\bfx}{\mbox{$\bf{x}$}}
\newcommand{\bfy}{\mbox{$\bf{y}$}}
\newcommand{\bfz}{\mbox{$\bf{z}$}}

\newcommand{\bsA}{\mbox{\boldmath $A$}}
\newcommand{\bsB}{\mbox{\boldmath $B$}}
\newcommand{\bsC}{\mbox{\boldmath $C$}}
\newcommand{\bsD}{\mbox{\boldmath $D$}}
\newcommand{\bsE}{\mbox{\boldmath $E$}}
\newcommand{\bsF}{\mbox{\boldmath $F$}}
\newcommand{\bsG}{\mbox{\boldmath $G$}}
\newcommand{\bsH}{\mbox{\boldmath $H$}}
\newcommand{\bsI}{\mbox{\boldmath $I$}}
\newcommand{\bsJ}{\mbox{\boldmath $J$}}
\newcommand{\bsK}{\mbox{\boldmath $K$}}
\newcommand{\bsL}{\mbox{\boldmath $L$}}
\newcommand{\bsM}{\mbox{\boldmath $M$}}
\newcommand{\bsN}{\mbox{\boldmath $N$}}
\newcommand{\bsO}{\mbox{\boldmath $O$}}
\newcommand{\bsP}{\mbox{\boldmath $P$}}
\newcommand{\bsQ}{\mbox{\boldmath $Q$}}
\newcommand{\bsR}{\mbox{\boldmath $R$}}
\newcommand{\bsS}{\mbox{\boldmath $S$}}
\newcommand{\bsT}{\mbox{\boldmath $T$}}
\newcommand{\bsU}{\mbox{\boldmath $U$}}
\newcommand{\bsV}{\mbox{\boldmath $V$}}
\newcommand{\bsW}{\mbox{\boldmath $W$}}
\newcommand{\bsX}{\mbox{\boldmath $X$}}
\newcommand{\bsY}{\mbox{\boldmath $Y$}}
\newcommand{\bsZ}{\mbox{\boldmath $Z$}}

\newcommand{\bfA}{\mbox{$\bf{A}$}}
\newcommand{\bfB}{\mbox{$\bf{B}$}}
\newcommand{\bfC}{\mbox{$\bf{C}$}}
\newcommand{\bfD}{\mbox{$\bf{D}$}}
\newcommand{\bfE}{\mbox{$\bf{E}$}}
\newcommand{\bfF}{\mbox{$\bf{F}$}}
\newcommand{\bfG}{\mbox{$\bf{G}$}}
\newcommand{\bfH}{\mbox{$\bf{H}$}}
\newcommand{\bfI}{\mbox{$\bf{I}$}}
\newcommand{\bfJ}{\mbox{$\bf{J}$}}
\newcommand{\bfK}{\mbox{$\bf{K}$}}
\newcommand{\bfL}{\mbox{$\bf{L}$}}
\newcommand{\bfM}{\mbox{$\bf{M}$}}
\newcommand{\bfN}{\mbox{$\bf{N}$}}
\newcommand{\bfO}{\mbox{$\bf{O}$}}
\newcommand{\bfP}{\mbox{$\bf{P}$}}
\newcommand{\bfQ}{\mbox{$\bf{Q}$}}
\newcommand{\bfR}{\mbox{$\bf{R}$}}
\newcommand{\bfS}{\mbox{$\bf{S}$}}
\newcommand{\bfT}{\mbox{$\bf{T}$}}
\newcommand{\bfU}{\mbox{$\bf{U}$}}
\newcommand{\bfV}{\mbox{$\bf{V}$}}
\newcommand{\bfW}{\mbox{$\bf{W}$}}
\newcommand{\bfX}{\mbox{$\bf{X}$}}
\newcommand{\bfY}{\mbox{$\bf{Y}$}}
\newcommand{\bfZ}{\mbox{$\bf{Z}$}}

\DeclareMathOperator*{\argmin}{argmin} 
\DeclareMathOperator*{\argmax}{argmax} 
\renewcommand{\footnotesize}{\fontsize{9pt}{11pt}\Large}

\usepackage[svgnames]{xcolor}
\usepackage{listings}

\lstset{language=R,
    basicstyle=\LARGE\tt,
    stringstyle=\color{DarkGreen},
    otherkeywords={0,1,2,3,4,5,6,7,8,9},
    morekeywords={TRUE,FALSE},
    deletekeywords={data,frame,length,as,character},
    %keywordstyle=\color{blue},
    commentstyle=\color{DarkGreen},
}
\CJKscale{0.9}
\begin{document}
\section{논의점1}
\para{problem setting} $\{X_\ell\}$을 눈이 이동한 자취라고 해석하자. 예를들어 눈을 3번 쌓아서 ($\tau=3$) 아래와 같이 눈이 이동하였다고 한다면 
\[
v_1 \to v_{20} \to v_{5}
\]
$X_1,X_2,X_3$은 아래와 같이 정의할 수 있다. 
\[
X_1=v_1, \quad X_2=v_{20}, \quad X_{3}=v_5
\]
$h_(v,\ell)$을 노드 $v$에서 $\ell$번째 눈을 쌓았을때 얻어지는 지형이라고 하자. $\ell=0$일 경우에는 $h(v,0)=f(v)$라고 정의하자. $X_\ell$과 $h(v,\ell)$의 묶음을 아래와 같이 정의하자. 
\[
\bar{\bfX}_\ell=(h(v_1,\ell),\dots,h(v_n,\ell),X_\ell)
\]
(여기에서 bar는 extended의 의미로 사용하였다.) 이 확률벡터는 아래의 공간에서 정의됨을 기억하자. 
\[
S=\mathbb{R}^n \times {\cal V}
\]
여기에서 $n$은 노드들의 총수이고, ${\cal V}=\{v_1,\dots,v_n\}$은 노드들의 집합이다.

\para{example} ${\cal V}=\{v_1,v_2\}$ 이고 
\[
f(v_1)=3, \quad f(v_2)=4
\]
라고 하자. $X_0=v_2$이고 눈이 쌓이는 양 $b=0.5$라고 하자. 이 경우 $X_1=v_1$이 된다. 그리고 
\[
\bar{\bfX}_0=(3,4,v_2)
\]
이고 
\[
\bar{\bfX}_1=(3.5,4,v_1)
\]
와 같이 표현할 수 있다. 
\ck 
확률벡터의 열 
\[
\bar{\bfX}_1, \bar{\bfX}_2, \dots
\]
는 틀림없이 (homogeneous) 마코프체인이다. 따라서 적당한 transition probability $p:(S,{\cal S})\to [0,1]$를 설정할 수 있다. 
이때 $p$는 transition probability의 정의에 따라서 아래의 성질을 만족한다. 
\begin{itemize}
	\item[(i)] 모든 $x \in S$에 대하여 $A \to p(x,A)$가 probability measure on $(S,{\cal S})$
	\item[(ii)] 모든 $A \in {\cal S}$에 대하여 $x \to p(x,A)$가 measurable function. 
\end{itemize} 
이럴때 $\{\bar{\bfX}_\ell\}$는 전이확률 $p$를 가지는 마코프체인이라고 표현한다. (w.r.t. ${\cal F}_\ell=\sigma(\bar{\bfX}_0,\dots,\bar{\bfX}_\ell)$)

\ck 그런데 $\bar{\bfX}_\ell$은 모든 state가 transient하다. 왜냐하면 눈이 쌓일수록 $h(v,\ell)$은 점점 증가하기 때문에 $\bar{\bfX}_\ell$이 머물러있는 state를 다시 방문할 수 없기 때문이다. 위의 예제로 들면 $\bar{\bfX}_0$가 머물러있는 state
\[
\bar{\bfX}_0=(3,4,v_2)
\]
를 모든 $\bar{\bfX}_1,\bar{\bfX}_2,\dots$들이 다시 방문할 수 없다. (눈은 점점 쌓이기만 하니까) 이는 에르고디시티와 같은 일부 유용한 성질들을 증명할 수 없다는 것과 같다. 따라서 우리는 $\bar{\bfX}_\ell$에 관심을 두지 않고 $X_\ell$에 관심을 둘것이다. 

\ck $\bar{\bfX}_\ell$이 정의되는 공간은 $S=\mathbb{R}^n\times {\cal V}$이며 state의 수가 무한개이지만 $X_\ell$은 ${\cal V}$에서 정의되며 state의 수가 유한개인 $n$이다. 어떠한 (homogeneous) 마코프체인이 finite state를 가지고 irreducible하면 에르고드하므로 $X_\ell$에 관심을 둔다면 에르고디시티와 같은 성질들을 보일 수 있을것 같다. 물론 $X_\ell$은 homogeneous하지 않으므로 위의 정리를 곧바로 적용할 순 없다. 

\ck \textcolor{red}{$X_\ell$은 전이확률행렬 $\bfP(\ell)$이 일정하지 않고 $\ell$에 따라 달라지므로 non-homogeneous한 마코프체인으로 해석할 수 있다고 생각하였다.} 다행이 Hajnal의 연구에 따르면 적당한 조건하에서 $X_\ell$이 에르고드함을 보일 수 있다. 

\ck 문제는 $X_\ell$이 전이확률행렬 $\bfP(\ell)$를 가지는 마코프체인이라고 볼 수 있는지 그렇지 않은지 헷갈린다는 점이다. 헷갈리는 이유는 아래와 같다. 
\begin{itemize}
	\item[(1)] $X_\ell$이 정의되는 상태집합은 ${\cal V}$인데 $\bfP(\ell)$이 정의되는 공간은 $(S,{\cal S})=(\mathbb{R}^n\times {\cal V}, {\cal S})$이다. 왜냐하면 지형의 모양을 알아야 다음확률을 결정할 수 있기 때문이다. 
	\item[(2)] 하지만 혹시 아래와 같이 표현할수도 있지 않을까? 하는 생각이 든다. \newline\newline
	$\bfP(\ell)$ as non-homogeneous transition matrix and $X_0,X_1,\dots,X_\ell$ as non-homogeneous Markov chain w.r.t ${\cal F}_\ell$ with $\bfP(\ell)$.
\end{itemize}

\ck 문제의 핵심은 어떠한 확률벡터가 마코프체인일경우 그 벡터의 일부원소를 취하면 그것을 마코프체인으로 해석할 수 있느냐는 것이다. 

\ck 질문을 바꾸어 표현하면 다음과 같다. 보통 $X_n$이 마코프체인일 경우 임의의 measurable function $f:\Omega_o \to \mathbb{R}$에 대하여 아래와 같은 strong markov property가 성립함이 알려져 있다. 
\[
E\Big({f(X_0,\dots,X_{n+1})} | {\cal F}_n \Big)=E_{X_n}f(X_0,\dots,X_{n+1})
\]
여기에서 $\Omega_o=S^{\{0,1,2,\dots,\}}$이다. 
이 성질을 우리의 문제상황에 맞추어 다시쓰면 아래와 같다. 
\[
E\Big({f(\bar{\bfX}_0,\dots,\bar{\bfX}_{\ell+1})} | {\cal F}_\ell \Big)=E_{\bar{\bfX}_\ell}f(\bar{\bfX}_0,\dots,\bar{\bfX}_{\ell+1})
\]
아래와 같이 함수 $f$를 정의한다면 
\begin{align*}
& f(\bar{\bfX}_0,\dots,\bar{\bfX}_{\ell+1}) \in S^{\{0\}} \times \dots \times S^{\{\ell\}} \times \big(\mathbb{R}^n \times V^*\big) \times \dots \times S^{\{\ell+2\}}\times \dots  \\ 
& \Longleftrightarrow 1(X_\ell\in V^*)
\end{align*}
아래와 같이 쓸 수 있다. 
\[
P\Big(X_{\ell+1} \in V^*| {\cal F}_\ell \Big)=P_{\bar{\bfX}_\ell}\Big(X_{\ell+1}\in V^*\Big)= P\Big(X_{\ell+1}\in V^* ~|~ {\bar{\bfX}_\ell}\Big)
\]
위의 식이 성립한다면 
\newline\newline
$X_0,X_1,\dots,X_\ell$ as non-homogeneous Markov chain w.r.t ${\cal F}_\ell$ with $\bfP(\ell)$.
\newline\newline
라고 표현할 수 있을까? 

\section{논의점2}
\ck Hajnal 1958의 정리3을 적용하려면 반드시 $\{X_\ell\}$이 마코프체인임을 보여야 하는 것일까? $\bfP(\ell)$이 단지 stochastic matrix\footnote{row-sum이 1인 매트릭스}이어도 충분하지 않을까? 

\section{논의점3}
\ck Hajnal 1958의 정리3에서 $n$대신에 stopping time을 넣어도 성립하는가? 

\end{document}

